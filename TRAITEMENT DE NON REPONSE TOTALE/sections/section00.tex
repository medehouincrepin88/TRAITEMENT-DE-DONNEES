\section{Introduction} % Seções são adicionadas para organizar sua apresentação em blocos discretos, todas as seções e subseções são automaticamente exibidas no índice como uma visão geral da apresentação, mas NÃO são exibidas como slides separados.

%------------------------------------------------
\begin{frame}{}
	
	\huge \begin{center}
		INTRODUCTION
	\end{center}
	
\end{frame}

\begin{frame}
	\frametitle{Introduction}
  


On dit qu'il y a non-réponse vis-à-vis de la variable Y pour l'individu échantillonné i dès lors que l'on ne dispose pas de la valeur $Y_i$ relative à cet individu. \\ \vspace{0.5cm}
 On distingue deux types de non réponse :
  \begin{enumerate}
  	\item les non réponses totales 
  	\item les non réponses partielles
  \end{enumerate}
 
\end{frame}


\begin{frame}
	\frametitle{Introduction}
	
La \textbf{non-réponse totale} est habituellement traitée par une méthode de \textbf{repondération}: \vspace{0.5cm}
	
\begin{enumerate}
\item	on supprime du fichier les non-répondants totaux,
\item	on augmente les poids des répondants pour compenser de la non-réponse totale.
\end{enumerate} 

\vspace{0.5cm}

La non-réponse partielle est habituellement traitée par imputation $\Rightarrow$
une valeur manquante est remplacée par une valeur plausible.


\end{frame}

\begin{frame}
	\frametitle{Introduction}

\textbf{L’objectif prioritaire} est de \textbf{réduire} autant que possible \textbf{le biais de
non-réponse} : cela passe par une recherche des facteurs explicatifs de la non-réponse

\end{frame}


%------------------------------------------------

%------------------------------------------------

\begin{frame}
	\frametitle{Non réponses totales }
	
On a une non réponse totale lorsque l’on n’a aucune donnée sur l’unité d’observation. \\ \vspace{0.5cm}

Autrement dit, on a aucune réponse aux questions posées. Cela ne signifie pas que l'on ne dispose aucune information sur le non-répondant : \\  \vspace{0.5cm}

En général, on dispose tout de même de renseignements présents dans la base de sondage, ou collectés sur le terrain (Par exemple des renseignements obtenus auprès de tierces personnes)


    %\begin{itemize}
     %   \item Lorem ipsum dolor sit amet.
      %  \item Lorem ipsum dolor sit amet.
    %\end{itemize}
	
\end{frame}



\begin{frame}
	\frametitle{Non réponses totales }
On va traiter ce problème par \textbf{repondération} : on fait porter aux répondants le poids des non-répondants. Cette repondération se justifie sous une modélisation du mécanisme de non-réponse.\\ \vspace{0.5cm}
Cette modélisation permet d’estimer les probabilités de réponse à
l’enquête, pour obtenir les poids corrigés de la non-réponse totale.
\end{frame}

\begin{frame}
	\frametitle{Quelques facteurs de non-réponse totale (Haziza, 2011)}
	
\begin{itemize}
	\item Mauvaise qualité de la base de sondage;
	\item Impossibilité de joindre l’individu;
	\item Type d’enquête (obligatoire ou volontaire);
	\item Fardeau de réponse;
	\item Méthode de collecte (interview, téléphone, courrier, ...);
	\item Durée de collecte;
	\item Suivi (et relance) des non-répondants;
    \item Formation des enquêteurs.
\end{itemize}

\end{frame}



\begin{frame}
	\frametitle{Hors champs}
	
Un point important : la distinction entre individus hors-champ et individus non-répondants \\ \vspace{0.4cm}


On parle de hors champs quand une valeur manquante est dû au fait que l’enquêté n’est pas concerné par la question : \\ \vspace{0.4cm}

\begin{enumerate}
	\item S'il concerne toute les variables, elle est total (HCT) 
	\item Ou partiel s’il concerne quelques variables (HCP) 
\end{enumerate} 

 \vspace{0.3cm}
La non réponses modifie les poids de sondage alors que le hors champs ne les modifie pas.
\end{frame}








