\section{Conclusion} % Seções são adicionadas para organizar sua apresentação em blocos discretos, todas as seções e subseções são automaticamente exibidas no índice como uma visão geral da apresentação, mas NÃO são exibidas como slides separados.

\begin{frame}{}
	
	\huge \begin{center}
		Conclusion
	\end{center}
	
\end{frame}

\begin{frame}{Conclusion}
  
  En conclusion, les méthodes de repondération face aux non-réponses sont cruciales pour garantir la fiabilité des résultats d'enquête. La repondération uniforme, le mécanisme de réponse homogène, et l'estimation des probabilités de réponse permettent de corriger les biais liés aux non-réponses.\\ \vspace{0.4cm}
  
   Chacune présente des avantages et des limites selon le contexte. Leur choix doit être adapté aux spécificités de l’enquête et de la population cible. \\ \vspace{0.4cm}
  
  Une approche bien choisie améliore la représentativité des données et la qualité des analyses.
\end{frame}



\begin{frame}{Références}
	\begin{enumerate}
\item Ardilly, Pascal. 2006. Les Techniques de Sondage / Pascal Ardilly,... Éditions Technip. https://bibliotheque.univ-catholille.fr/Default/doc/SYRACUSE/340736/les-techniques-de-sondage-pascal-ardilly.\\ \vspace{0.5cm}
\item Chauvet, Guillaume. n.d. “Données Manquantes dans les Enquêtes.”
	\end{enumerate}


\end{frame}